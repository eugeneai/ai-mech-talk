\documentclass[10pt]{beamer}

\usepackage{iftex,ifxetex}
\ifPDFTeX
  \usepackage[utf8]{inputenc}
  \usepackage[T1]{fontenc}
  \usepackage[russian]{babel}
  \usepackage{lmodern}
  \usefonttheme{serif}
\else
  \ifluatex
    \usepackage{unicode-math}
    \defaultfontfeatures{Ligatures=TeX,Numbers=OldStyle}
    \setmathfont{Latin Modern Math}
    \setsansfont{Linux Biolinum O}
    \setmonofont{Fira Code}
    \usefonttheme{professionalfonts}
    % \setmathfont[
    %     Ligatures=TeX,
    %     Scale=MatchLowercase,
    %     math-style=upright,
    %     vargreek-shape=unicode
    %     ]{euler.otf}
  \fi
\fi


\usepackage{amsmath,amssymb,longtable,hhline}
\usepackage{mathrsfs}
\usepackage{xcolor}
\usepackage{listings}
\usepackage{hyperref}
\usepackage{multicol}
\usepackage{anyfontsize}
\usepackage{minted}
\usepackage{tikz}
\usepackage{indentfirst}
%\usepackage{enumitem}

\usetikzlibrary{matrix}
% \setlist[description]{leftmargin=0pt,labelindent=\parindent}

\usemintedstyle{tango}
\newcommand{\ltprgsize}{\fontsize{5}{5}\selectfont}
\setminted{fontsize=\ltprgsize,mathescape}

\definecolor{mygreen}{rgb}{0,0.6,0}
\definecolor{mygray}{rgb}{0.5,0.5,0.5}
\definecolor{mymauve}{rgb}{0.58,0,0.82}

\hypersetup{
    bookmarks=true,         % show bookmarks bar?
    unicode=true,           % non-Latin characters in Acrobat’s bookmarks
    pdftoolbar=false,        % show Acrobat’s toolbar?
    pdfmenubar=false,        % show Acrobat’s menu?
    pdffitwindow=false,     % window fit to page when opened
    pdfstartview={FitH},    % fits the width of the page to the window
    pdftitle={Компьютерная алгебра в задачах оптимизации},    % title
    pdfauthor={Evgeny Cherkashin, Seseg Badmatsyrenova},     % author
    pdfsubject={symbolic computations},   % subject of the document
    pdfnewwindow=true,      % links in new PDF window
    colorlinks=true,       % false: boxed links; true: colored links
    linkcolor=red,          % color of internal links (change box color with linkbordercolor)
    citecolor=green,        % color of links to bibliography
    filecolor=magenta,      % color of file links
    urlcolor=blue           % color of external links
}

\usepackage{pifont}

\usetheme{Warsaw}
\usecolortheme{crane}
%\useinnertheme{rectangles}
\setbeamertemplate{itemize item}{\scriptsize\hbox{\donotcoloroutermaths\ding{113}}}
\setbeamertemplate{itemize subitem}{\tiny\raise1.5pt\hbox{\donotcoloroutermaths$\blacktriangleright$}}
\setbeamertemplate{itemize subsubitem}{\tiny\raise1.5pt\hbox{\donotcoloroutermaths$\blacktriangleright$}}
\setbeamertemplate{enumerate item}{\insertenumlabel.}
\setbeamertemplate{enumerate subitem}{\insertenumlabel.\insertsubenumlabel}
\setbeamertemplate{enumerate subsubitem}{\insertenumlabel.\insertsubenumlabel.\insertsubsubenumlabel}
\setbeamertemplate{enumerate mini template}{\insertenumlabel}

\beamertemplatenavigationsymbolsempty


%\useoutertheme{split}
%\useinnertheme{rounded}
\setbeamertemplate{background canvas}[vertical shading][bottom=white!80!cyan!20,top=cyan!10]
%\setbeamertemplate{sidebar canvas left}[horizontal shading][left=white!40!black,right=black]

\graphicspath{{pics/}}


% --------------------------

% Продукт (Конкуренты, свойства)
% Графический аспект
% Технические аспекты
% Содержание сайта
% Дополнительная информация
% Контактная информация


  \tikzset{
    table/.style={
        matrix of nodes,
        row sep=-\pgflinewidth,
        column sep=-\pgflinewidth,
        nodes={
            rectangle,
            draw=black,
            align=center
        },
        minimum height=1.5em,
        text depth=0.5ex,
        text height=2ex,
        nodes in empty cells,
%%
        every even row/.style={
            nodes={fill=gray!20}
        },
        column 1/.style={
 %           nodes={text width=2em,font=\bfseries}
            nodes={font=\bfseries}
        },
        row 1/.style={
            nodes={
                % fill=black,
                % text=white,
                font=\bfseries
            }
        }
    }
}

% \makeatletter
% \define@key{beamerframe}{t}[true]{% top
%   \beamer@frametopskip=.2cm plus .5\paperheight\relax%
%   \beamer@framebottomskip=0pt plus 1fill\relax%
%   \beamer@frametopskipautobreak=\beamer@frametopskip\relax%
%   \beamer@framebottomskipautobreak=\beamer@framebottomskip\relax%
%   \def\beamer@initfirstlineunskip{}%
% }
% \makeatother

\newcommand{\Tr}{\mbox{True}}
\newcommand{\Fa}{\mbox{False}}


\begin{document}
\setlength{\parindent}{1em}
\title{Механизмы искусственного интеллекта}

\author{Е.~А.~Черкашин}
\institute[ИДСТУ СО РАН]{ИДСТУ им.~В.~М.~Матросова СО РАН}
\date[10.02.2025]{{}\\[1.5cm]
%<<>>\\
10 февраля 2025~г., Иркутск
}
%\date{\today}
\maketitle

\begin{frame}[fragile]
  \frametitle{К определению понятия <<Искусственный интеллект>>}

  К ИИ относятся программы, реализующие задачи, решение которых потребовало бы от человека использования его \emph{когнитивных способностей}, выполнения \emph{творческих функций}\footnote{Справочник ИИ, Т.1, 1990}.

  \vspace{1em}

  Определение из книги AIMA\footnote{P.Norvig, S.Russell. Artificial Intelligence Modern Approach}:
  \begin{center}
    \begin{tikzpicture}
      \matrix (first) [table,text width=6em]
      {
        &\!Рационально & Как человек \\
        Рассуждать   & $\bigcirc$ & $\bigcirc$ \\
        Вести себя   & $\bigcirc$ & $\bigcirc$ \\
      };
    \end{tikzpicture}
  \end{center}
\end{frame}

\begin{frame}
  \frametitle{Тест Тьюринга}
  \begin{center}
    \includegraphics[width=0.9\linewidth]{pics/turing-test.pdf}
  \end{center}

  Если Человек в процессе общения с ИИ будет думать, что общается с Человеком, то ИИ обладает требуемым свойством.
\end{frame}


\begin{frame}
  \frametitle{ИИ как модель человека}
  \begin{center}
    Моделирование рассуждений \to{} моделирование <<математики>>\\
    \includegraphics[width=0.2\linewidth]{pics/ai-approach.pdf}\\
    Моделирование мозга \to{} моделирование\\ самоорганизующихся систем
  \end{center}
%   \textbf{ИИ как математическая модель}
\end{frame}

\begin{frame}
  \frametitle{Свойства и классы задач ИИ}
  \small
\begin{columns}
  \begin{column}{0.4\textwidth}
    \textbf{Свойства задач}
    \begin{itemize}
    \item Не существует алгоритма решения
    \item Обработка символьной информации
    \item Автоматизация принятия решения
    \item Решение -- комбинация <<возможностей>>
    \item Обработка неполной и противоречивой информации
    \end{itemize}
    \vspace{6.5em}
    \mbox{}
  \end{column}
  \begin{column}{0.6\textwidth}
    \textbf{Классы задач}
    \begin{enumerate}
    \item Игры
    \item Автоматическое доказательство теорем
    \item Восприятие и распознавание образов
    \item Планирование действий (Problem Solving)
    \item Понимание естественного языка, перевод
    \item Логическое программирование
    \item Экспертные системы
    \item Интеллектные и аналитические информационные системы
    \item Восприятие и усвоение знаний (Machine learning)
    \item Интеллектное управление [производством] \ldots
    \item Робототехника (Robotics)
    \item Системы поддержки принятия решений
    \end{enumerate}
  \end{column}
\end{columns}
\end{frame}

\begin{frame}
  \frametitle{Планирование действий (Problem Solving)}

\begin{columns}
  \begin{column}{0.4\textwidth}
    \includegraphics[width=1\linewidth]{pics/boxes.png}
  \end{column}
  \begin{column}{0.4\textwidth}
    \includegraphics[width=1\linewidth]{pics/boxes-state-space.png}
  \end{column}
\end{columns}
\vfill
\textbf{Формализация} ($SSG$, State Space Graph)
\begin{itemize}
\item Структура данных для представления \textbf{состояния}
\item Формализация \textbf{правил перехода} из состояния в состояние $G$
\item Программирование \textbf{распознавателя \emph{целевого состояния}} $R(v)$
\item Задание \textbf{исходного состояния} $I$
\end{itemize}
\vspace{1em}
\[
  SSG = <G, R(v), I>, G=<V,E>, v\in V.
\]
\end{frame}

% \begin{frame}
%   \frametitle{Игры}
% \end{frame}

\begin{frame}[fragile]
  \frametitle{Представление знаний}
\small
Естественно определить данные как некоторые сведения об отдельных объектах, а знания~-- о мире в целом.

{\bf Данные} представляют информацию о существовании объектов, представляемых значениями признаков, а {\bf знания}~-- информацию о существующих в мире закономерных связях между признаками и запрещающих некоторые другие сочетания свойств у объектов.

{\bf Данные}~-- это информация о существовании объектов с некоторым набором свойств, {\bf знания}~-- информация о \textbf{не}существовании объектов с некоторым набором свойств.

Пусть $H(x)$ $\Leftrightarrow$ <<$x$~является человеком>>, а $M(x)$~-- <<$x$~-- смертен>>. \\

\noindent{}<<сократ, $s$, является человеком>>:
$$
    A_1=H(s),
$$
<<не существует бессмертных людей>>):
$$
    A_2=\neg\exists x \big ( H(x) \& \neg M(x) \big ),
$$
преобразуем во <<все люди смертны>>:
$$
    A_3=\forall x \big ( H(x)\to M(x)\big ).
$$
\ldots все объекты $x$, обладающие свойством $H$, будут обладать свойством $M$.

\end{frame}

\begin{frame}[fragile]
  \frametitle{Обработка информации <<в математике>>}
\small
Добавим информатики. Утверждение <<Сократ~-- человек>> представим как $H(s)$, где $s$ -- это Сократ.  Теперь из \textbf{знания} $A_3$ и \textbf{исходного данного} $H(s)$ получим новую информацию о Сократе:
$$
B = \Big (H(s) \& \overbrace{\forall x \big ( H(x)\to M(x)\big )}^{A_3} \Big ) \to M(s).
$$

Из того, что Сократ~-- человек и что все люди смертны, следует, что Сократ тоже смертен $M(s)$.  Доказательство <<от противного>>:

\begin{enumerate}
\item Требуется \textbf{опровергнуть высказывание} $\neg B = \Tr$
\item $B = \Fa \vdash$ $H(s) \& A_3 = \Tr$, $M(s) = \Fa$
\item $H(s) \&  A_3 = \Tr \vdash H(s) = \Tr$, {\color{red}$A_3 = \Tr$} \label{pa}
\item Подставим $s$ вместо $x$ в $A_3$, получим $H(s)\to M(s)$
\item $H(s)= \Tr, M(s)= \Fa \vdash H(s)\to M(s) = \Fa$
\item $H(s)\to M(s) = \Fa\vdash$ {\color{blue} $A_3 = \Fa$} \label{pb}
\item Из п.~\ref{pa} следует {\color{red}$A_3 = \Tr$}, а из п.~\ref{pb} -- {\color{blue} $A_3 = \Fa$} \label{contr}
\item В формальной (математической) логике в п.~\ref{contr} получено \textbf{противоречие}
\item Следовательно, $\neg B \neq \Fa$, $B = \Tr$ \label{stmt}
\item Из п.~\ref{stmt} следует $M(s)=\Tr$ $\square$\ \ \ (\textbf{ЧТД, QED\footnote{\emph{Quod Erat Demonstrandum}, <<что и требовалось показать>>, ``which was to be demonstrated``})}
% \begin{align*}
% 2x - 5y &=  8 \\
% 3x + 9y &=  -12
% \end{align*}
\end{enumerate}
\end{frame}

\begin{frame}[fragile]
  \frametitle{Логическое программирование}

  \textbf{Сократим доказательство}

  Будем выписывать только \textbf{истинные высказывания}, т.е.
  $M(s) \Leftrightarrow M(s) = \Tr$ (<<Сократ смертен>>),\\
  $\neg M(s) \Leftrightarrow M(s) = \Fa$ (<<Сократ бессмертен>>)

$$
B = \Big (H(s) \& \overbrace{\forall x \big ( H(x)\to M(x)\big )}^{A_3} \Big ) \to M(s).
$$

\begin{columns}
  \begin{column}{0.5\textwidth}\footnotesize
    \begin{enumerate}
    \item \textbf{Опровергаем} $\neg B$
    \item $B\vdash$ $H(s) \& A_3$, $M(s)$
    \item $H(s) \& A_3 \vdash H(s)$, {\color{red}$A_3$} \label{pa}
    \item $A_3\{s/x\} = H(s)\to M(s)$
    \item $H(s), \neg M(s) \vdash \neg (H(s)\to M(s))$
    \item $\neg(H(s)\to M(s))\vdash$ {\color{blue} $\neg A_3$} \label{pb}
    \item Из п.~\ref{pa} следует {\color{red}$A_3$}, а из п.~\ref{pb} -- {\color{blue} $\neg A_3$} \label{contr}
    \item В п.~\ref{contr} получено \textbf{противоречие}
    \item Следовательно, $B$ \label{stmt}
    \item Из п.~\ref{stmt} $M(s)$ $\square$
    \end{enumerate}
    \vspace{2.7em}
    \mbox{}
  \end{column}
  \begin{column}{0.5\textwidth}
    \textbf{\footnotesize Программа (теория)}
\begin{minted}{prolog}
h(s).
m(X) :- h(X).   % h(x) -> m(x).
\end{minted}
    \textbf{\footnotesize Консультация}
\begin{minted}{text}
?- [socrates].  % Загрузка теории из файла
true.

?- m(s).        % Запрос
true.
\end{minted}
\textbf{\footnotesize Объяснение вывода}
\begin{minted}{text}
?- trace.
true.

[trace]  ?- m(s).
   Call: (10) m(s) ?     % Запрос пользователя
   Call: (11) h(s) ?     % Применение m(X):-h(X)
                         % при {s/X}
   Exit: (11) h(s) ?     % Найден факт h(s).
   Exit: (10) m(s) ?     % следовательно m(s).
true.

[trace]  ?-
\end{minted}
  \end{column}
\end{columns}
\end{frame}

\begin{frame}[fragile]
  \frametitle{Планирование действий}
  \begin{center}
    \includegraphics[width=0.7\linewidth]{pics/maze.pdf}
  \end{center}
\begin{columns}
  \begin{column}{0.5\textwidth}
  \textbf{Теория лабиринта}
\begin{minted}{prolog}

% "Данные"

e(a,b). e(b,c). e(c,f).
e(f,e). e(e,d). e(d,h).
e(f,g). e(g,i). e(i,h).

% "Знания"
% Что такое "путь".

path(A,B) :- e(A,B).
path(A,B) :- e(A,C), path(C,B).

\end{minted}

  \noindent\textbf{Запрос}

\begin{minted}{text}
?- path(a,h).
true ;
true ;
false.
\end{minted}
\end{column}
\begin{column}{0.5\textwidth}
\textbf{Конструктивное решение}
\begin{minted}{prolog}
path(A,B, [A-B]) :- e(A,B).
path(A,B, [A-C|T]) :- e(A,C), path(C,B,T).
\end{minted}
\textbf{Запрос}
\begin{minted}{text}
?- path(a,h,L).
L = [a-b, b-c, c-f, f-e, e-d, d-h] ;
L = [a-b, b-c, c-f, f-g, g-i, i-h] ;
false.
\end{minted}
\vspace{4em}
\mbox{}
\end{column}
\end{columns}
\end{frame}

\begin{frame}[fragile]
  \frametitle{Стратегии поиск в SSG}
  \begin{center}\tiny
    \def\svgwidth{0.8\linewidth}
    \def\za{A}
    \def\zb{B}
    \def\zc{C}
    \def\ya{a)}
    \def\yb{b)}
    \def\yc{c)}
    \input{pics/searchdescr.pdf_tex}
  \end{center}
\begin{columns}
  \begin{column}{0.5\textwidth}
    \textbf{Поиск в глубину}
\begin{minted}{prolog}
dfs(V,[]):- r(V).
dfs(V,[V-N|T]):- \+ r(V), after(V,N), dfs(N,T).
r(h).
after(X,Y):- e(X,Y); e(Y,X).
\end{minted}
\begin{minted}{text}
?- dfs(a, S).
S = [a-b, b-c, c-f, f-e, e-d, d-h] ;
S = [a-b, b-c, c-f, f-e, e-d, d-e, e-d, d-h] ;
S = [a-b, b-c, c-f, f-e, e-d, d-e, e-d, d-e, ...]
\end{minted}
    %\vspace{1em}
    \mbox{}
  \end{column}
  \begin{column}{0.5\textwidth}
    \textbf{Поиск в ширину}
\begin{minted}{prolog}
bfs([[X|T]|_],[X|T]):-r(X),!.   % (1)
bfs([[X|T]|Ways], S):-
    findall([Y,X|T],            % (2)
      (after(X,Y), \+ member(Y,[X|T])),
    L),
    append(Ways, L, NWays),     % (3)
    bfs(NWays, S).
\end{minted}
\begin{minted}{text}
?- bfs([[a]],S).
S = [h, d, e, f, c, b, a].
\end{minted}
  \end{column}
\end{columns}
\end{frame}

\begin{frame}[fragile]
  \frametitle{Использование дополнительной информации}
  \begin{columns}
    \begin{column}{0.5\textwidth}\footnotesize
      Зададим функцию $f\!\!:V\to R$ следующего вида:
\begin{equation*}
  % \label{eq:1}
  f(x)=g(x)+r(x)
\end{equation*}
-- стоимость кратчайшего пути (КП) через $x$,  $g(x)$~-- до $x$, $r(x)$~-- КП до целевой вершины (не известен).

Оценка $r(x)$ снизу -- $h(x)$, $h(x)\leqslant r(x)$.  Теперь
\begin{equation*}
  %\label{eq:2}
  f(x)\leqslant g(x)+h(x).
\end{equation*}
    \end{column}
    \begin{column}{0.5\textwidth}
\begin{minted}{prolog}
bf1([_-s(G,[Target|T]) |_],% (1)
    Target-_,s(G,[Target|T])):-!.
bf1([_-s(G,[X|T])|Ways], Target-GPS, S):-
    Target\=X,
    findall(F1-s(G1,[Y,X|T]),
      after([X|T],G,GPS, Y,G1,F1), L),
    append(L, Ways, NWays), keysort(NWays,SNWays),
    bf1(SNWays,Target-GPS,S).
after([S|R],SG, GPS, T,TG, F):-
    transdist(S,T,D), % \+ member(T,[S|R]),
    TG is SG + D,
    geodist(T, GPS, GDist), % (2)
    F is TG + GDist.
bf(Start, Target, Sol):-
    geocode(Target, Lon, Lat, _),
    bf1([0-s(0,[Start])],
        Target-ll(Lon,Lat), Sol). % (3)
\end{minted}
    \end{column}
  \end{columns}
  \begin{center}
    \includegraphics[width=0.7\linewidth]{pics/yuzd1.pdf}
  \end{center}
\end{frame}

\begin{frame}[fragile]
  \frametitle{Игра 15}
  \begin{columns}
    \begin{column}{0.5\textwidth}
      \includegraphics[width=1\linewidth]{pics/15-puzzle_magical.pdf}
    \end{column}
    \begin{column}{0.5\textwidth}
\textbf{Поиск решения с использованием эвристики}
\begin{minted}{text}
center% ./15-solve 100 20 1
Environment:
USE_HEURISTIC=1
Puzzle 15 solving program
State(00,00[00])
<---solution--->
0 Step ------------
 5  1  3  4
 9  2  6  8
13 10 15  0
14 12  7 11
x=2, y=3
State(00,16[16])
. . . . . .
17 Step ------------
 1  2  3  4
 5  6  7  8
 9 10 11 12
13 14 15  0
x=3, y=3
State(17,00[17])
Tested 111 states.
\end{minted}
\textbf{Без эвристики}
\begin{minted}{text}
. . . . . .
17 Step ------------
 1  2  3  4
 5  6  7  8
 9 10 11 12
13 14 15  0
x=3, y=3
State(17,00[17])
Tested 921299 states.
\end{minted}
    \end{column}
  \end{columns}
\end{frame}

% \begin{frame}
%   \frametitle{Понимания естественного языка: актуальность}
%   \footnotesize
%   Понимание естественного языка (ЕЯ), перевод из одного ЕЯ на другой -- \textbf{Направление ИИ}. Решаются следующие задачи:
%   \begin{enumerate}
%   \item Анализ текстов, помещение изъятой информации в базу данных:
%     \begin{itemize}
%     \item изготовление шаблонов документов, отчетов;
%     \item синтез структур данных для ИС;
%     \item заполнение баз данных ИС и т.п.;
%     \end{itemize}
%   \item Ведение диалога с пользователем:
%     \begin{itemize}
%     \item идентификация моделей и планирование действий (интеллектные пользовательские интерфейсы);
%     \item приобретение знаний (оболочки экспертных систем);
%     \end{itemize}
%   \item Управление приложением:
%     \begin{itemize}
%     \item запросы на естественном языке к базам данных;
%     \item внесение изменений в данные;
%     \item командное управление (<<Проветрить квартиру>>).
%     \end{itemize}
%   \end{enumerate}
%   % Цель лекции -- рассмотреть методику использования ЕЯ в качестве языка запросов к базе данных.
% \end{frame}

% \begin{frame}
%   \frametitle{Графический пользовательский интерфейс}
%   Специализирован на \textbf{операциях над отдельными частями} информационного объекта.
%   \begin{center}
%     \includegraphics[width=0.4\linewidth]{pics/shot-context-menu.png}\qquad
%     \includegraphics[width=0.4\linewidth]{pics/shot-replace.png}
%   \end{center}
%   В операциях аргументы вводится в диалоговом окне (для операций с аргументами).

%   Более <<умные>> редакторы не используют контекстные операции (EMACS, VI, Visual Studio Code, Sublime, AutoCAD).
% \begin{quote}
%   \texttt{Alt-X replace-string}, Набирается как <<Alt-X repl str>>
% \end{quote}
% \begin{center}
%     \includegraphics[width=0.9\linewidth]{pics/shot-replace-string-emacs.png}\qquad
%   \end{center}
% \end{frame}

% \begin{frame}
%   \frametitle{Язык как математическая модель}
%   \begin{block}{Семиотика (наука о знаках) делится на три раздела (Моррис)}
%   \begin{itemize}
%   \item \textbf{Семантика} — отношение знака к объекту: \\ Что значит \emph{знак}?
%   \item \textbf{Синтаксис} — отношение знаков между собой --\\ как создаются \emph{новые смыслы} (термины, суждения) комбинированием \emph{знаков}.
%   \item \textbf{Прагматика} — отношение знака к субъекту:\\ Что \emph{обозначает предложение}, что надо дальше делать? На какой конкретно вопрос и как надо отвечать?
%   \end{itemize}
% \end{block}
% \end{frame}

% \begin{frame}
%   \frametitle{Язык программирования}
% \textbf{Синтаксис} на уровне грамматики определяет корректные последовательности символов (операторы, структуры). Но синтаксическая правильность не гарантирует даже осмысленности программы. % Таким образом, синтаксис определяет лишь одну сторону языка.

% \textbf{Семантика} — это соответствие между синтаксически правильными программами и [вариантами] действий абстрактного исполнителя, то есть это смысл синтаксических конструкций.

% \textbf{Прагматика} задает конкретизацию абстрактного вычислителя (конкретный процессор и др. ресурсы) для  вычислительной системы. Стандарт языка программирования задаёт поведение вычислителя не полностью, конкретный транслятор языка переводит программу в конкретной машинный код на конкретную программно-аппаратную платформу.

% Реализованный язык является прагматическим опосредованием абстрактной модели вычислений и ее реализацией на конкретном компьютере.
% \vfill
% \textbf{Цель программиста} — получить нужный ему эффект в результате исполнения программы на конкретном оборудовании: трансляция и исполнение осуществляется на конкретных вычислителях.

% \end{frame}

% \begin{frame}
%   \frametitle{Синтаксический разбор предложения}
%   \begin{center}
%     \includegraphics[width=1\linewidth]{pics/morphosent.pdf}
%   \end{center}
% \end{frame}

% \begin{frame}
%   \frametitle{Грамматика}
%    \begin{columns}
%      \begin{column}{0.6\textwidth}
%        \begin{block}{}
%        \[
%          G=\langle T,N,\Sigma,R\rangle
%        \]
%      \end{block}
%        $T$ -- множество терминальных символов (слова, буквы, \texttt{IF}, \texttt{ELSE}),\\
%        $N$ -- множество нетерминальных символов (обозначения, $\Alpha, \Beta$, <noun>, <verb>), $T\cap N=\emptyset,$\\
%        $\Sigma$ -- стартовый символ (<программа>, <предложение>), $\Sigma\in N,$\\
%        $R$ -- множество правил грамматики $\Alpha\to\Beta$, \\$R\subset((T\cup N)^* N(T \cup N)^*)\times(T\cup N)^*$.\\[0.5em]
%        Язык $L(G)=\{\Omega\in T^* | \Sigma\to^*\Omega\}$.
%      \end{column}
%      \begin{column}{0.4\textwidth}
%        \begin{block}{Вывод $\Sigma\to^*\Omega$}
%          \[\Sigma\to\Sigma\Alpha\quad \Sigma\to\Alpha\]
%          \[\Alpha\to b\Sigma e\quad \Alpha\to be \]
%          Пример: \(a=`(',\quad b=`)'\).
%          \begin{raggedleft}
%            \[
%              \begin{array}{ll}
%                \Sigma & \Sigma\\
%                \Alpha & \Alpha\\
%                b\Sigma e & ( \Sigma )\\
%                b\Sigma\Alpha e & ( \Sigma\Alpha )\\
%                b\Alpha\Alpha e & ( \Alpha\Alpha )\\
%                bbe\Alpha e & (() \Alpha )\\
%                bbebee & (()())
%              \end{array} \]
%          \end{raggedleft}
%      \end{block}
%      \end{column}
%    \end{columns}
% \end{frame}

% \begin{frame}
%   \frametitle{Типы грамматик}
%   По иерархии Ноама Хомского, грамматики делятся на \textbf{четыре} типа, каждый последующий является более ограниченным подмножеством предыдущего (но и легче поддающимся анализу):
%   \begin{itemize}
%     \item Тип 0. Неограниченные грамматики — возможны любые правила;
%     \item Тип 1. Контекстно-зависимые грамматики — левая часть может содержать один нетерминал, окруженный «контекстом»; сам нетерминал заменяется непустой последовательностью символов в правой части;
%     \item Тип 2. Контекстно-свободные грамматики — левая часть состоит из одного нетерминала;
%     \item Тип 3. Регулярные грамматики — более простые, распознаются конечными автоматами.
%   \end{itemize}
% \end{frame}

% \begin{frame}[fragile]
%   \frametitle{Грамматика языка программирования С}
%   {\ltprgsize
%  \begin{verbatim}
% <translation-unit> ::= {<external-declaration>}*

% <external-declaration> ::= <function-definition>
%                    | <declaration>

% <function-definition> ::= {<declaration-specifier>}* <declarator> {<declaration>}* <compound-statement>
%                    | union

% <struct-declaration> ::= {<specifier-qualifier>}* <struct-declarator-list>
% . . . . . .
% <specifier-qualifier> ::= <type-specifier>
%                    | <type-qualifier>

% <struct-declarator-list> ::= <struct-declarator>
%                    | <struct-declarator-list> , <struct-declarator>

% <struct-declarator> ::= <declarator>
%                    | <declarator> : <constant-expression>
%                    | : <constant-expression>
% . . . . . .
% <selection-statement> ::= if ( <expression> ) <statement>
%                    | if ( <expression> ) <statement> else <statement>
%                    | switch ( <expression> ) <statement>

% <iteration-statement> ::= while ( <expression> ) <statement>
%                    | do <statement> while ( <expression> ) ;
%                    | for ( {<expression>}? ; {<expression>}? ; {<expression>}? ) <statement>

% <jump-statement> ::= goto <identifier> ;
%                    | continue ;
%                    | break ;
%                    | return {<expression>}? ;
% \end{verbatim}}
% \end{frame}

% \begin{frame}[fragile]
%   \frametitle{Пример трансляции}
%    \begin{columns}
%      \begin{column}{0.5\textwidth}
%  \begin{minted}{c}
% #include <stdio.h>

% typedef unsigned long int ulint;

% ulint fact (ulint n) {
%     if (n==0) return 1;
%     if (n==1) return 1;
%     return n*fact(n-1);
% }

% int main() {
%     ulint n = 10;
%     printf("Factorial of %lu = %lu.\n",
%         n, fact(n));
%     return 0;
% }
%  \end{minted}
%      \end{column}
%       \begin{column}{0.5\textwidth}
% \begin{minted}{asm}
% 	.file   "fact.c"
% 	.text
% 	.globl  fact
% 	.type   fact, @function
% fact:
% .LFB11:
% 	movl    $1, %eax
% 	cmpq    $1, %rdi
% 	jbe     .L4
% .L3:
% 	imulq   %rdi, %rax
% 	subq    $1, %rdi
% 	cmpq    $1, %rdi
% 	jne    .L3
% .L4:
% 	ret
% .LFE11:
% 	.section    .rodata.str1.1,"aMS",@progbits,1
% .LC0:
% 	.string     "Factorial of %lu = %lu.\n"
% 	.section    .text.startup,"ax",@progbits
% 	.globl  main
% 	.type   main, @function
% main:
% .LFB12:
% 	;; . . . . . . . . . . .
% 	ret
% 	.cfi_endproc
% .LFE12:
% 	.size   main, .-main
% 	.ident  "GCC: (GNU) 8.2.1 20181127"
% 	.section    .note.GNU-stack,"",@progbits
% \end{minted}
%       \end{column}
%     \end{columns}
%  \end{frame}

% \begin{frame}[fragile]
%   \frametitle{Синтаксический разбор предложения}
%   Корова трясет хвостом.\\
%   A cow shakes the tail.

%   \verb|% [a, cow, shakes, the, tail]|

%   \textbf{Грамматика:}
%   \begin{itemize}
%   \item Множество \textbf{терминальных} символов -- \{a,b,c,\ldots,z\}.  На самом деле, $\Sigma=\{a,cow,shakes,walks,\ldots\}$.
%   \item Множество \textbf{нетерминальных} символов -- \texttt{<sentence>, <noun>, <verb>, \ldots}
%   \item \textnormal{Стартовый} символ -- \texttt{<sentence>}.
%   \item \textbf{Правила} упрощенного английского языка:
%     \begin{align}
%       \label{eq:eng-gramm}
%       \langle{}sentence\rangle{} & \to  \langle{}noun group\rangle{} \langle{}verb group\rangle{} \\
%       \langle{}noun group\rangle{} & \to  \langle{}determinant\rangle{} \langle{}noun\rangle{} \\
%       \langle{}verb group\rangle{} & \to  \langle{}verb\rangle{} \langle{}noun group\rangle{} \\
%       \langle{}noun\rangle{} & \to cow\ |\ tail\ |\ \ldots \\
%       \langle{}verb\rangle{} & \to walks\ |\ shakes\ |\ \ldots \\
%       \langle{}determinant\rangle{} & \to a\ |\ the\ |\ \varepsilon\ |\ my\ |\ \ldots
%     \end{align}
%   \end{itemize}
% \end{frame}

% \begin{frame}[fragile,fragile]
%   \frametitle{Разбор предложения на Prolog}
%   \begin{columns}
%   \begin{column}{0.5\textwidth}
% \begin{minted}{prolog}
% % <sent> ::== <noun group> <verb group>
% % <noun group> ::== <det> <noun>
% % <verb group> ::== <verb> <noun group>
% % <det> ::== a | the | my | yours | its
% % <noun> ::== cow | tail | body
% % <verb> ::== walks | shakes | moves

% % ?- t(sent, [a,cow,shakes,its,tail], [], Tree).

% noun(cow). noun(tail). noun(body).
% %noun(X):-member(X,[cow, tail, body]).
% det(X):-member(X,[a, the, my, yours, its]).
% verb(X):-member(X,[walks, shakes, moves]).

% t(sent, I, O, sent(NG,VG)):-
%     t(ng, I, R, NG),
%     t(vg, R, O, VG).
% t(ng, I, O, ng(Det,N)):-
%     t(det,I,R, Det),
%     t(noun,R,O,N).
% t(vg, I, O, vg(V,NG)):-
%     t(verb, I,R, V),
%     t(ng, R,O, NG).
% t(det, [X|I],I, det(X)):-
%     det(X).
% t(verb,[X|I],I, verb(X)):-
%     verb(X).
% t(noun, [X|I],I, noun(X)):-
%     noun(X).
% \end{minted}
% \begin{minted}{text}
% ?- [lp].
% true.

% ?- t(sent, [a,cow,shakes,its,tail], [], T).
% T = sent(ng(det(a), noun(cow)), vg(verb(shakes),
%          ng(det(its), noun(tail)))) ;
% false.
% \end{minted}
%   \end{column}
%   \begin{column}{0.5\textwidth}
%     \includegraphics[width=1\linewidth]{pics/cow-its-tail.png}
%   \end{column}
% \end{columns}
% \end{frame}

% \begin{frame}
%   \frametitle{Linked grammar (грамматики связей)}
%   \begin{columns}
%     \begin{column}{0.4\textwidth}
%       \includegraphics[width=0.9\linewidth]{pics/l-g-parts.png}
%       \includegraphics[width=0.9\linewidth]{pics/l-g-table.png}
%     \end{column}
%     \begin{column}{0.6\textwidth}
%       Последовательность слов находится в linked grammar, если существует способ нарисовать связи между словами, такие что\\[1em]
% %      Sequence of words is in the language of link grammar if there is a way to draw links between words in such a way that

%       Связи не пересекаются (\textbf{планарный граф});\\[0.5em]
%       Все слова последовательности соединены связями (\textbf{связность});\\[0.5em]
%       Все связи удовлетворяют ограничениям (\textbf{непротиворечивость})\\[0.5em]
%       Два слова соединены одной и только одной связью (\textbf{исключительность}).\\[0.5em]
%     \end{column}
%   \end{columns}
% \end{frame}

% \begin{frame}
%   \frametitle{Примеры разбора Linked grammar}
%   \centering
%   “The cat chased a snake”\\
%   \includegraphics[width=0.6\linewidth]{pics/l-g-good-ex.png}\\[1em]
%   “The Mary chased cat”\\
%   \includegraphics[width=0.6\linewidth]{pics/l-g-bad-ex.png}
% \end{frame}

% \begin{frame}
%   \frametitle{Примеры разбора Linked grammar}
%   \centering
%   “A dog arrived with a bone”\\
%   “A dog with a bone arrived”\\
%   \includegraphics[width=0.6\linewidth]{pics/l-g-multi-ex.png}\\[1em]
% \end{frame}

% \begin{frame}
%   \frametitle{Примеры разбора Linked grammar}
%   \centering
%   \includegraphics[width=0.8\linewidth]{pics/l-g-prog-ex.png}\\[1em]
% \end{frame}

% \begin{frame}[fragile,fragile]
%   \frametitle{Информационная система GeoBase. База данных}
%   GeoBase -- программа, позволяющая делать запросы на ЕЯ к базе данных по географии США, Borland, 1988.
% \begin{minted}{prolog}
% state('alabama','al','montgomery',3894e3,51.7e3,22,'birmingham','mobile','montgomery','huntsville').
% state('alaska','ak','juneau',401.8e3,591e3,49,'anchorage','fairbanks','juneau','sitka').

% city('alabama','al','birmingham',284413).
% city('alabama','al','mobile',200452).

% border('florida','fl',['georgia','alabama']).

% highlow('alabama','al','cheaha mountain',734,'gulf of mexico',0).

% mountain('alaska','ak','mckinley',6194).
% mountain('alaska','ak','st. elias',5489).

% road('66',['district of columbia','virginia']).

% lake('huron',59570,['michigan']).
% \end{minted}
%   {\ttfamily\ltprgsize
% \begin{verbatim}


%   The database contains the following information:

% Information about states:
%   Area of the state in square kilometers
%   Population of the state in citizens
%   Capital of the state
%   Which states border a given state
%   Rivers in the state
%   Cities in the state
%   Highest and lowest point in the state in meters

% Information about rivers:
%   Length of river in kilometers

% Information about cities:
%   Population of the city in citizens
% \end{verbatim}}
% \end{frame}

% \begin{frame}[fragile]
%    \frametitle{Примеры запросов}
% {\ttfamily\footnotesize
% \begin{verbatim}
% Some sample queries:

%    - states

%    - give me the cities in california.

%    - what is the biggest city in california ?

%    - what is the longest river in the usa?

%    - which rivers are longer than 1 thousand kilometers?

%    - what is the name of the state with the lowest point?

%    - which states border alabama?

%    - which rivers do not run through texas?

%    - which rivers run through states that border the state
%      with the capital austin?
% \end{verbatim}}
% \end{frame}

% \begin{frame}[fragile]
%   \frametitle{Схемы (спецификации) интерпретации}
%   \begin{columns}
%     \begin{column}{0.4\textwidth}
% \begin{minted}{prolog}
% schema('abbreviation','of','state').
% schema('state','with','abbreviation').
% schema('capital','of','state').
% schema('state','with','capital').
% schema('population','of','state').

% schema('area','of','state').
% schema('city','in','state').

% schema('length','of','river').
% schema('state','with','river').
% schema('river','in','state').

% schema('capital','with','population').
% schema('point','in','state').

% schema('height','of','point').
% schema('mountain','in','state').

% schema('height','of','mountain').
% schema('lake','in','state').

% schema('name','of','river').
% schema('name','of','capital').

% schema('road','in','continent').
% \end{minted}
%     \end{column}
%     \begin{column}{0.6\textwidth}
%       \includegraphics[width=1\linewidth]{pics/sem-net-1.pdf}
%     \end{column}
%   \end{columns}
% \end{frame}

% \begin{frame}[fragile]
%   \frametitle{Программа}
%   Основной цикл программы
% \begin{minted}{prolog}
% geobase(STR, X, E):-
%         STR \= "",
%         atom_string(ATOM,STR),
%         tokenize_atom(ATOM,LIST),     /* Returns a list of words(symbols)           */
%         filter(LIST,LIST1),           /* Removes punctuation and words to be ignored*/
%         pars(LIST1,E,Q),              /* Parses queries                            */
%         findall(A,eval_interp(Q,A),L),
%         unik(L,L1),
%         % unit(E,U),
%         member(X,L1).
% \end{minted}
%   Синтаксический анализатор (часть)
% \begin{minted}{prolog}
%   pars(LIST,E,Q):-s_attr(LIST,OL,E,Q),check(OL),!.
%   pars(LIST,_,_):-error(LIST),fail.
% % . . . . . .
%   /* How big is the biggest city -- BIG QUERY */
%   s_attr([BIG|S1],S2,E1,q_eaq(E1,A,E2,Q)):-
% 		size(_,BIG),s_minmax(S1,S2,E2,Q),
% 		size(E2,BIG),entitysize(E2,E1),
% 		schema(E1,A,E2),!.

%   s_attr(S1,S2,E,Q):-s_minmax(S1,S2,E,Q).
% % . . . . . .

% /* ... the shortest river in texas -- MIN QUERY */
%   s_assoc1([MIN|S1],S2,E1,A,q_eaq(E1,A,E2,q_min(E2,Q))):-minn(MIN),!,
% 		s_nest(S1,S2,E2,Q),schema(E1,A,E2).

% /* ... the longest river in texas -- MAX QUERY */
%   s_assoc1([MAX|S1],S2,E1,A,q_eaq(E1,A,E2,q_max(E2,Q))):-maxx(MAX),!,
% 		s_nest(S1,S2,E2,Q),schema(E1,A,E2).
% \end{minted}
% \end{frame}

% \begin{frame}[fragile]
%   \frametitle{Корпус, источник данных}
%   \texttt{population of Washington}\\
%   Население штата или города?\\
%   \textbf{?-} \texttt{schema('population','of','city').}\\
%   \textbf{?-} \texttt{schema('population','of','state').}\\
%   Корпус реализован при помощи реструктуризации базы данных.
% \begin{minted}{prolog}
%   % . . . . . . . . .
%   /* Relationships about states */
%   db(abbreviation,of,state,ABBREVIATION,STATE):-  state(STATE,ABBREVIATION,_,_,_,_,_,_,_,_).
%   db(state,with,abbreviation,STATE,ABBREVIATION):-state(STATE,ABBREVIATION,_,_,_,_,_,_,_,_).
%   db(area,of,state,AREA,STATE):-        state(STATE,_,_,_,AREA1,_,_,_,_,_),str_real(AREA,AREA1).
%   db(capital,of,state,CAPITAL,STATE):-  state(STATE,_,CAPITAL,_,_,_,_,_,_,_).
%   db(state,with,capital,STATE,CAPITAL):-state(STATE,_,CAPITAL,_,_,_,_,_,_,_).
%   db(population,of,state,POPULATION,STATE):-state(STATE,_,_,POPUL,_,_,_,_,_,_),str_real(POPULATION,POPUL).
%   db(state,border,state,STATE1,STATE2):-border(STATE2,_,LIST),member(STATE1,LIST).

%   /* Relationships about rivers */
%   db(length,of,river,LENGTH,RIVER):-    river(RIVER,LENGTH1,_),str_real(LENGTH,LENGTH1).
%   db(state,with,river,STATE,RIVER):-    river(RIVER,_,LIST),member(STATE,LIST).
%   db(river,in,state,RIVER,STATE):-      river(RIVER,_,LIST),member(STATE,LIST).

%   /* Relationships about points */
%   db(point,in,state,POINT,STATE):-      highlow(STATE,_,POINT,_,_,_).
%   db(point,in,state,POINT,STATE):-      highlow(STATE,_,_,_,POINT,_).
%   db(state,with,point,STATE,POINT):-    highlow(STATE,_,POINT,_,_,_).
%   db(state,with,point,STATE,POINT):-    highlow(STATE,_,_,_,POINT,_).
%   db(height,of,point,HEIGHT,POINT):-    highlow(_,_,_,_,POINT,H),str_real(HEIGHT,H),!.
%   db(height,of,point,HEIGHT,POINT):-    highlow(_,_,POINT,H,_,_),str_real(HEIGHT,H),!.
%   % . . . . . . . . .
% \end{minted}
% \end{frame}

% \begin{frame}[fragile]
%   \frametitle{Программа}
%   Интерпретация запроса \texttt{\ldots findall(A,eval\_interp(Q,A),L), \ldots}
% \begin{minted}{prolog}
%   eval_interp(Q,IAns):-
% 		eval(Q,A),
% 		e_i(A,IAns).

%   eval(q_min(ENT,TREE),ANS):-
% 		findall(X,eval(TREE,X),L),
% 		entitysize(ENT,ATTR),
% 		sel_min(ENT,ATTR,99e99,'',ANS,L).

%   eval(q_max(ENT,TREE),ANS):-
% 		findall(X,eval(TREE,X),L),
% 		entitysize(ENT,ATTR),
% 		sel_max(ENT,ATTR,-1,'',ANS,L).

%   eval(q_sel(E,gt,ATTR,VAL),ANS):-
% 		schema(ATTR,ASSOC,E),
% 		db(ATTR,ASSOC,E,SVAL2,ANS),
% 		str_real(SVAL2,VAL2),
% 		VAL2>VAL.
% % . . . . . . .
%   eval(q_eaq(E1,A,E2,TREE),ANS):-
% 		eval(TREE,VAL),db(E1,A,E2,ANS,VAL).

%   eval(q_eaec(E1,A,E2,C),ANS):-db(E1,A,E2,ANS,C).

%   eval(q_e(E),ANS):-	ent(E,ANS). % EVAL "ATOM"

%   eval(q_or(TREE,_),ANS):- eval(TREE,ANS).

%   eval(q_or(_,TREE),ANS):- eval(TREE,ANS).

%   eval(q_and(T1,T2),ANS):- eval(T1,ANS1),eval(T2,ANS),ANS=ANS1.
% \end{minted}
% \end{frame}

% \begin{frame}[fragile]
%   \frametitle{Пример запуска программы Geobase}
%   \footnotesize
% \begin{minted}{text}
% Welcome to SWI-Prolog (threaded, 64 bits, version 9.0.4)
% SWI-Prolog comes with ABSOLUTELY NO WARRANTY. This is free software.
% Please run ?- license. for legal details.

% For online help and background, visit https://www.swi-prolog.org
% For built-in help, use ?- help(Topic). or ?- apropos(Word).

% ?- [geobase].
% true.

% ?- loaddba.
% Loading database file - please wait
% true.

% ?- geobase:geobase("states").
% alabama alaska arizona arkansas california
% colorado connecticut delaware florida georgia
% hawaii idaho illinois indiana iowa
% kansas kentucky louisiana maine maryland
% massachusetts michigan minnesota mississippi missouri
% montana nebraska nevada new hampshire new jersey
% new mexico new york north carolina north dakota
% ohio oklahoma oregon pennsylvania rhode island
% south carolina south dakota tennessee texas
% utah vermont virginia washington west virginia
% wisconsin wyoming

% 50 Solutions
% true.

% ?- geobase:geobase("which rivers run through states that border the state with the capital austin?").
% neosho washita arkansas st. francis white
% mississippi ouachita pearl red canadian
% cimarron rio grande san juan gila pecos

% 15 Solutions
% true.
% \end{minted}
% \end{frame}

\begin{frame}
  \frametitle{Chat GPT 4}
  ChatGPT (Generative Pretrained Transformer, Порождающий [пред]треннированный преобразователь) -- чат-бот OpenAI, запущенный в ноябре 2022.

  \ldots настроен при помощи обучения <<с учителем>> и <<с подкреплением>>.  Именно обучение с подкреплением делает ChatGPT \textbf{уникальным}.

  \ldots предназначен для реагирования на входные данные, представленные на естественном языке.

  \begin{center}
    \includegraphics[width=0.7\linewidth]{pics/transformer-model-architecture.png}
  \end{center}
\end{frame}

\begin{frame}
  \frametitle{Нейронные сети}
  \begin{center}
    \includegraphics[width=1\linewidth]{pics/nn-all.pdf}
  \end{center}
\end{frame}

\begin{frame}
  \frametitle{Анализ входного текста}
  \begin{center}
    \includegraphics[width=1\linewidth]{pics/gpt-parsing.png}
  \end{center}
\end{frame}

\begin{frame}
  \frametitle{Воспитание нейронной сети}
  \begin{center}
    \includegraphics[width=1\linewidth]{pics/ChatGPT_Diagram.pdf}
  \end{center}
\end{frame}

\begin{frame}
  \frametitle{Цель воспитания}
  \begin{center}
    \includegraphics[width=1\linewidth]{pics/capability-versus-alignment.png}
  \end{center}
\end{frame}

% \begin{frame}
%   \frametitle{QR-код презентации}
%   \centering
%   \includegraphics[width=0.5\linewidth]{pics/qr-code.png}
% \end{frame}

\end{document}

% talk-iccs-de-lect-2023-07.tex

%%% Local Variables:
%%% mode: latex
%%% TeX-master: t
%%% End:
